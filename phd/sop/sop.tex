\documentclass[12pt]{article}
\usepackage[utf8]{inputenc}
\setlength{\oddsidemargin}{0in}
\setlength{\evensidemargin}{0in}
\setlength{\textwidth}{6.5in}
\setlength{\topmargin}{-.3in}
\setlength{\textheight}{9in}
\pagestyle{empty}

\begin{document}

\begin{center}
{\Large Statement of Purpose} \\[.3in]
{\large Adam Kosiorek}
\end{center}

\vspace*{.5in}

I like challenges. After reading an article about genetic engineering I have chosen Biology as my major in the secondary school. With the biggest challange I faced being memorization of some facts, I decided to change my major to Maths \& Physics I always loved. I would have chosen Informatics if only I had any idea what that meant. Coming from a background where CS is associated with technicians repairing “broken” PCs, I had an idea of CS very far from the romantic vision of young startup founders set out to change the world. Luckily, I have been exposed to Matrix and IRobot, both filled with autonomous intelligent machines ruleing the world. Fascinated, I decied to find out if it is possible and to how make it happen. Not that I wanted to destroy the world, but the sheer concept of creating a machine that could improve itself, or technological singularity, name I didn't know at the time, was enticing. 

My journey began at the best tech university in the country, where I enrolled for the BSc in Robotics. I learned a lot about physics and mechanics, material science and manufacturing technologies; not until my first programming class, however, have I realized how a truly intelligent entinty could be controled. This, paired with my passion for vision and perception I've developed as a photojournalist steered me towards my first computer vision course, where I learned about image processing and machine learning. I immedietely knew that this was the area I wanted to focus on. It was not only challangeing, since there weren't any general purpose systems good enough for deployment in the real world, but it was also well aligned for my original passion for intelligent machines.

I also try to learn from the best. That's why I went fn or an internship to the Samsung R\&D Lab in Warsaw, where I've been working on object classification in the Computer Vision group. I've been exposed there to the Bag of Words approach for classification a-nd consequently, to the state-of-the-art in keypoint detection and feature description, which helped me to later understand why representations learnd automatically by convolutional neural networks are essential. I focused mainly on improving the quantization step of BoW approach – that is building visual vocabulary. I took it a step further and implemented a similar system for 3D point cloud classifications with the Microsoft Kinect as my bachelor thesis. I continued my work at Samsung where I was introduced to deep learning.

My observation was that deep learning required huge computational resources to be any good and therefore I decided to do CSE@TUM to be able to harness supercomputers in favour of my research. I've been working with some algorithms developed there and it made it look like a good idea. The prof' Cremers lab further increased attractivity. So did the DAAD scholarship, which meant I could focus on my studies. 

This summer I did an internship at Bloomberg in London, where I've been working on fraud detection in financial transactions. The problem can be cast as unsupervised anomaly detection with further verification in a supervised settings. I learned how difficult it can be to introduce an innovative approach in a corporate setting. This, together with my earlier industrial experiences conviced me that I do want to pursue a PhD. I love solving scientific problems that do not have “the best” solution or reference specification and I vastly prefer it to chiseling icons in a UI. 

Currently I'm working with Caner Hazirbas and Rudolph Triebel of the Computer Vision group headed by Prof. Cremers. Our recent project is the introspective capacity of neural networks. Introspective capacity is understood as the ability of a classifier to assess uncertainty of its result given the data [ref]. It is an important problem, because in mobile robotics or medical settings a wrong classification might lead to the lose of life. While high classification accuracy is certainly desired, it is even more important to know how certain classification results are. Our intermediate results show that neural networks, augmented by additional layers and a novel cost function, can be jointly trained for classification and uncertainty estimation. The topic might expand into my master thesis, which I am intending to write during the summer semester.

In my PhD research project I would like to focus on deep neural networks for reinforcement learning and recurrent neural networks, topics that I do not have much experience in. The first problem fascinates me in that it seems to be similar to how humans learn. It is far easier to specify a reward function or to designate an expert as in the inverse version of the problem than to provide supervised training data. This means that a bulk of reinforcement learning can be done in the unsupervised fashion, which is, I believe, THE way of bringing the cutting edge AI technology to the masses. RNNs, on the other hand, are well suited for processing sequential information. As such, this model might contribute to decreasing the bulk of computation required e.g. in object classification in videos. One way of doing that would be to introduce beliefs about what might be in the image in the form of priors computed from the preceeding elements of the sequence. Another interesting use case would be optical flow computation from a sequence of images. The only end-to-end neural network for optical flow computation, while efficient and accurate, requires that the consecutive images of a video sequence are input in pairs. I would like to investigate an RNN designs capale of computing optical flow while presented with a single image at a time.

Passion for solving technical problems coupled with my demonstrated skills show that I am ready and extremely well motivated to take part in the Traction Europe workshop. I am confident that I will be able to contribute significantly to and benefit immensely from the workshop. Thank you for considering my application.
\end{document}


% 1. What's Needed.
% 
% General area of research
% Relevant skills and experience
% Your reasons for applying to Oxford
% Why you consider yourself a strong applicant.
% 
% 2. Introspection.
% 
% Intelectual Development.
% 
% a. Matrix and IRobot hooked me up on AI and Robotics
% b. Found B.Sc. in Robotics at WUT and went for it
% c. Disapointed due to mechanical engineering oriented programme
% d. Course in CV and mobile robotics -> that’s where AI is, should do CS
% e. Internship in Samsung in CV: ML is really cool, need more CS skills
% f. Thesis in Object Recognition
% g. 1 semester of CS undegrad courses
% h. CV/ML needs HPC for large scale -> CSE with focus on CV/ML
% i. Want to do CV/ML for robotics as PhD -> RL
% 
% Role Models: Andrew Ng and Elon Musk.
% 
% Both extremely creative, a scientist and an applied engineer, both with huge vision. One is changing the world of AI/ML and education, the other is chaining the world by reaching for technologies far out of time. Bold, don’t fear to lose, have a dream and want to change the world. They don’t care about problems being too hard or too big. They just make stuff happen.
% 
% 3. Outline.
% 
% 1. Genesis
% 1. I always liked physics and maths
% 2. As a photographer I was interested in vision and perception
% 3. I was fascinated by robotics and AI, although I had no idea what that even was
% 
% 2. What Happened
% 1. Went to study Robotics @ WUT
% 2. Turned out that most of Robotics is mechanical and electrical engineering with a nudge of control theory. I got more excited about general software development and got into it UNTIL my first class in computer vision where I learned about machine learning.
% 3. Decided I want to do this and went to the Samsung R&D Lab to work objcect classification and was doing bachelor thesis on the same topic but in 3D with kinect
% 4. Turnened out that TUM is quite strong in the area and I decided to go for a Masters hers, DAAD helped a lot
% 5. I'm working with the Prof. Cremers' group on deep learning for computer vision. I'm really into it and I want to get a PhD. I love solving difficult research problems that have no right or wrong solution. I hate being tied up by the specs and working on that tiny little widget to make the UI better.
% 
% 3. Outlook
% 1. Introspective capacity is vital for robotics. I'm looking into it in the context of deep learning and I'd like to make it work, but it's not sth I want to do a PhD on; Masters might be good
% 2. Reinforcement Learning fascinates me in that it is, together with the inverse problem, more or less how we learn. I'd like to study it further for I believe it is THE way of bringing ML to the masses. It's way easier to specify a reward  function or to find an expert than to gather supervised training datasets. I'm interested into human learning processes as well and I wonder if we can port them to the neural network settings
% 3. Recurrent neural networks seem suitable to a plethora of various tasks. E.g. object classification, and even more object detection, seem to be doing a lot of redundant work when dealing with e.g. video sequences. I'd like to work on itroducing knowledge of what might be in the sequence as priors based on what we've already seen there. Another thing is to use RNNs to compute optical flow from a sequence, but using one image at a time. Also RNNs for optical flow from DVS would be cool. 
