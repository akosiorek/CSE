\documentclass[12pt]{article}
\usepackage[utf8]{inputenc}
\setlength{\oddsidemargin}{0in}
\setlength{\evensidemargin}{0in}
\setlength{\textwidth}{6.5in}
\setlength{\topmargin}{-.3in}
\setlength{\textheight}{9in}
\pagestyle{empty}

\setlength{\parindent}{0em}
\setlength{\parskip}{1em}

\begin{document}

\begin{center}
{\Large Statement of Purpose} \\[.3in]
{\large Adam Kosiorek}
\end{center}

\vspace*{.5in}

%The Statement of Purpose should describe succinctly your reasons for applying to the proposed %program at Stanford, your preparation for this field of study, research interests, future career %plans, and other aspects of your background and interests which may aid the admissions committee %in evaluating your aptitude and motivation for graduate study. The Statement of Purpose should not %exceed two pages.

Artificial Intelligence has always fascinated me. Long before understanding what computer science stands for I wondered why AI has not been achieved yet. To explore it further, I enrolled in BSc in Robotics at the best technical university in my country. Soon after, it became apparent that pursuing a PhD was one of the few ways that could get me closer to the answer.

Robotics is far more than AI - or so I would learn during the first few years of undergrad. I studied maths, physics and mechanics, material science and manufacturing technologies; not until my first programming class, however, did I realize how a truly intelligent entity could be controlled. Paired with my passion for vision and perception, which I have developed as a photojournalist, it steered me towards my first computer vision course. I immediately knew that this intersection of image processing and machine learning was the area I wanted to focus on. Without any systems good enough for real world deployment, $it was challenging as well as aligned with my original passion for intelligent machines. $

To learn from the best, I joined Computer Vision Lab at Samsung R\&D in Warsaw for an internship. My project was to design and implement a Bag of Words-based pipeline for object classification. Consequently, I learned about the state-of-the-art in keypoint detection and feature description, which later helped me to appreciate the importance of automatically learned low level features in convolutional neural networks (CNNs). I focused my research mainly on the visual codebook generation within Bag of Words. The final step was to optimize it and then port it to the Android platform. In my bachelor thesis I investigated how spatial information affects classification performance by implementing a similar system for Kinect-gathered point cloud classification. Afterwards, I continued my work at Samsung, where I turned to deep learning techniques for object classification and image duplicate detection.

My observation was that deep learning, although often superior to older approaches, required huge computational resources and datasets. To mitigate this problem I went to study computational science at the Technical University of Munich. With cutting-edge computer vision research groups it seemed a great choice. Numerical linear algebra and scientific computing, comprising the majority of the coursework, help enormously with optimization and implementation of algorithms. Currently, I am working with Caner Hazirbas and Rudolph Triebel of the Computer Vision group headed by Prof. Cremers. In our recent project we investigate the introspective capacity of neural networks. It is understood as the ability of a classifier to assess uncertainty of its prediction given the data \cite{introspective}. It is a paramount problem in mobile robotics and medicine where wrong classification might lead to loss of life. While high classification accuracy is desired, it is impossible to assess whether a given prediction is accurate in a test setting, where no labels are available. It is, therefore, vital to assess the uncertainty of predictions. Our intermediate results show that neural networks, augmented with additional layers and a novel cost function, can be jointly trained for classification and uncertainty estimation. The topic might expand into my master thesis.

This summer I did an internship at Bloomberg in London, where I worked on fraud detection in financial transactions. The problem can be cast as unsupervised anomaly detection with further verification in a supervised settings. I learned how difficult it can be to introduce an innovative approach in a corporate setting. This, together with my earlier industrial experience, convinced me that I do want to pursue a PhD. I love solving scientific problems, which do not have ``the best'' solution or a reference specification, by going into the deepest details.

In my doctoral research project I would like to focus on deep neural networks for reinforcement learning and recurrent neural networks. The former fascinates me since it is similar to how humans learn. Specifying a reward function  or designating an expert in the inverse problem  is far easier than providing supervised training data. Recurrent neural networks, on the other hand, are well suited to sequential information processing. If beliefs about content could be introduced in a form of priors inferred from previous elements of the sequence, it might be possible to increase object classification accuracy in videos and to shrink the network size. Another interesting problem is an end-to-end RNN for optical flow computation. The only end-to-end neural network, while efficient and accurate, does not use temporal information. It computes optical flow for any two possibly unrelated images \cite{flownet}. I would like to investigate RNN designs capable of computing optical flow while presented with a single image at a time. Solving both problems might thus lead to smaller networks and reduced amounts of computation, possibly increasing energy efficiency of such systems.

My passion for solving technical problems coupled with my demonstrated skills show that I am ready and extremely motivated to carry out research as a PhD student at the University of Oxford. I am confident that I will be able to contribute significantly to and benefit immensely from my stay in Oxford. Thank you for considering my application.

\bibliographystyle{ieeetr}
\bibliography{bibliography}
\end{document}

% 1. What's Needed.
% 
% General area of research
% Relevant skills and experience
% Your reasons for applying to Oxford
% Why you consider yourself a strong applicant.
% 
% 2. Introspection.
% 
% Intelectual Development.
% 
% a. Matrix and IRobot hooked me up on AI and Robotics
% b. Found B.Sc. in Robotics at WUT and went for it
% c. Disapointed due to mechanical engineering oriented programme
% d. Course in CV and mobile robotics -> that’s where AI is, should do CS
% e. Internship in Samsung in CV: ML is really cool, need more CS skills
% f. Thesis in Object Recognition
% g. 1 semester of CS undegrad courses
% h. CV/ML needs HPC for large scale -> CSE with focus on CV/ML
% i. Want to do CV/ML for robotics as PhD -> RL
% 
% Role Models: Andrew Ng and Elon Musk.
% 
% Both extremely creative, a scientist and an applied engineer, both with huge vision. One is changing the world of AI/ML and education, the other is chaining the world by reaching for technologies far out of time. Bold, don’t fear to lose, have a dream and want to change the world. They don’t care about problems being too hard or too big. They just make stuff happen.
% 
% 3. Outline.
% 
% 1. Genesis
% 1. I always liked physics and maths
% 2. As a photographer I was interested in vision and perception
% 3. I was fascinated by robotics and AI, although I had no idea what that even was
% 
% 2. What Happened
% 1. Went to study Robotics @ WUT
% 2. Turned out that most of Robotics is mechanical and electrical engineering with a nudge of control theory. I got more excited about general software development and got into it UNTIL my first class in computer vision where I learned about machine learning.
% 3. Decided I want to do this and went to the Samsung R&D Lab to work objcect classification and was doing bachelor thesis on the same topic but in 3D with kinect
% 4. Turnened out that TUM is quite strong in the area and I decided to go for a Masters hers, DAAD helped a lot
% 5. I'm working with the Prof. Cremers' group on deep learning for computer vision. I'm really into it and I want to get a PhD. I love solving difficult research problems that have no right or wrong solution. I hate being tied up by the specs and working on that tiny little widget to make the UI better.
% 
% 3. Outlook
% 1. Introspective capacity is vital for robotics. I'm looking into it in the context of deep learning and I'd like to make it work, but it's not sth I want to do a PhD on; Masters might be good
% 2. Reinforcement Learning fascinates me in that it is, together with the inverse problem, more or less how we learn. I'd like to study it further for I believe it is THE way of bringing ML to the masses. It's way easier to specify a reward  function or to find an expert than to gather supervised training datasets. I'm interested into human learning processes as well and I wonder if we can port them to the neural network settings
% 3. Recurrent neural networks seem suitable to a plethora of various tasks. E.g. object classification, and even more object detection, seem to be doing a lot of redundant work when dealing with e.g. video sequences. I'd like to work on itroducing knowledge of what might be in the sequence as priors based on what we've already seen there. Another thing is to use RNNs to compute optical flow from a sequence, but using one image at a time. Also RNNs for optical flow from DVS would be cool. 
