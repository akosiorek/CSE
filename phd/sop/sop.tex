\documentclass[12pt]{article}
\usepackage[utf8]{inputenc}
\setlength{\oddsidemargin}{0in}
\setlength{\evensidemargin}{0in}
\setlength{\textwidth}{6.5in}
\setlength{\topmargin}{-.3in}
\setlength{\textheight}{9in}
\pagestyle{empty}

\begin{document}

\begin{center}
{\Large Statement of Purpose} \\[.3in]
{\large Adam Kosiorek}
\end{center}

\vspace*{.5in}

I like challenges. After reading an article about genetic engineering I have chosen Biology as my major in the secondary school. With the biggest challenge I faced being memorization of some facts, I decided to change my major to Maths \& Physics I always loved. I would have chosen Informatics if only I had any idea what that meant. Coming from a background where CS is associated with technicians repairing “broken” PCs, my idea of CS was very far from the romantic vision of young startup founders set out to change the world. Luckily, I have been exposed to Matrix and IRobot, both filled with autonomous intelligent machines ruling the world. Fascinated, I decided to find out if it is possible and how make it happen. Not that I wanted to destroy the world, no, but the sheer concept of creating a machine that could improve itself, or the technological singularity, name I didn't know at the time, was enticing. 

My journey began at the best tech university in the country, where I enrolled for the BSc in Robotics. I learned a lot about physics and mechanics, material science and manufacturing technologies; not until my first programming class, however, have I realized how a truly intelligent entity could be controlled. This, paired with my passion for vision and perception I've developed as a photojournalist, steered me towards my first computer vision course, where I learned about image processing and machine learning. I immediately knew that this was the area I wanted to focus on. It was not only challenging, since there weren't any general purpose systems good enough for deployment in the real world, but it was also well aligned with my original passion for intelligent machines.

To learn from the best, I went for an internship to the Computer Vision Lab at Samsung R\&D in Warsaw. My project was to design and implement a Bag of Words-based pipeline for object classification. Consequently, I learned about the state-of-the-art in keypoint detection and feature description, which later helped me to appreciate the importance of automatically learned low level features in convolutional neural networks (CNNs). I focused my research mainly on the visual codebook generation within Bag of Words. The final step was to optimize it and port to the Android platform. In my bachelor thesis I investigated how spatial information affect classification performance by implementing a similar system for Kinect-gathered point cloud classification. I continued my work at Samsung, where I worked on deep learning techniques for object classification and duplicate detection.

My observation was that deep learning, although superior to older approaches, required huge computational resources and datasets to be any good and therefore I decided to study computational science to be able to harness supercomputers. TUM seemed to be a great choice due to the computer vision teams operating there. 

This summer I did an internship at Bloomberg in London, where I've been working on fraud detection in financial transactions. The problem can be cast as unsupervised anomaly detection with further verification in a supervised settings. I learned how difficult it can be to introduce an innovative approach in a corporate setting. This, together with my earlier industrial experience, convinced me that I do want to pursue a PhD. I love solving scientific problems that do not have ``the best'' solution or reference specification and I vastly prefer it to chiseling icons in an user interface. 

Currently, I am working with Caner Hazirbas and Rudolph Triebel of the Computer Vision group headed by Prof. Cremers. In our recent project we investigate the introspective capacity of neural networks. It is understood as the ability of a classifier to assess uncertainty of its prediction given the data [ref]. It is a paramount problem in mobile robotics or medicine where a wrong classification might lead to loss of life. While high classification accuracy is desired, it is impossible to assess whether a given prediction is accurate in the test setting, where no labels are available. It is, therefore, vital to assess uncertainty of predictions. Our intermediate results show that neural networks, augmented with additional layers and a novel cost function, can be jointly trained for classification and uncertainty estimation. The topic might expand into my master thesis.

In my doctoral research project I would like to focus on deep neural networks for reinforcement learning and recurrent neural networks. The former fascinates me since it is similar to how humans learn. Specifying a reward function  or, as in the inverse problem, designating an expert is far easier than providing supervised training data. Recurrent neural networks, on the other hand, are well suited to sequential information processing. If beliefs about content could be introduced in a form of priors inferred from previous elements of the sequence, it might be possible to increase object classification accuracy in videos and to shrink the network size. It would reduce the amount of computation needed, possibly leading to an energy efficient system for portable devices. Another interesting problem is an end-to-end RNN for optical flow computation. The only end-to-end neural network, while efficient and accurate, does not use temporal information. It computes optical flow for any two possibly unrelated images [ref]. I would like to investigate RNN designs capable of computing optical flow while presented with a single image at a time. It might lead to better accuracy with fewer computations, thus better energy efficiency.

Passion for solving technical problems coupled with my demonstrated skills show that I am ready and extremely well motivated to carry out research as a PhD student at the University of Oxford. I am confident that I will be able to contribute significantly to and benefit immensely from my stay there. Thank you for considering my application.
\end{document}


% 1. What's Needed.
% 
% General area of research
% Relevant skills and experience
% Your reasons for applying to Oxford
% Why you consider yourself a strong applicant.
% 
% 2. Introspection.
% 
% Intelectual Development.
% 
% a. Matrix and IRobot hooked me up on AI and Robotics
% b. Found B.Sc. in Robotics at WUT and went for it
% c. Disapointed due to mechanical engineering oriented programme
% d. Course in CV and mobile robotics -> that’s where AI is, should do CS
% e. Internship in Samsung in CV: ML is really cool, need more CS skills
% f. Thesis in Object Recognition
% g. 1 semester of CS undegrad courses
% h. CV/ML needs HPC for large scale -> CSE with focus on CV/ML
% i. Want to do CV/ML for robotics as PhD -> RL
% 
% Role Models: Andrew Ng and Elon Musk.
% 
% Both extremely creative, a scientist and an applied engineer, both with huge vision. One is changing the world of AI/ML and education, the other is chaining the world by reaching for technologies far out of time. Bold, don’t fear to lose, have a dream and want to change the world. They don’t care about problems being too hard or too big. They just make stuff happen.
% 
% 3. Outline.
% 
% 1. Genesis
% 1. I always liked physics and maths
% 2. As a photographer I was interested in vision and perception
% 3. I was fascinated by robotics and AI, although I had no idea what that even was
% 
% 2. What Happened
% 1. Went to study Robotics @ WUT
% 2. Turned out that most of Robotics is mechanical and electrical engineering with a nudge of control theory. I got more excited about general software development and got into it UNTIL my first class in computer vision where I learned about machine learning.
% 3. Decided I want to do this and went to the Samsung R&D Lab to work objcect classification and was doing bachelor thesis on the same topic but in 3D with kinect
% 4. Turnened out that TUM is quite strong in the area and I decided to go for a Masters hers, DAAD helped a lot
% 5. I'm working with the Prof. Cremers' group on deep learning for computer vision. I'm really into it and I want to get a PhD. I love solving difficult research problems that have no right or wrong solution. I hate being tied up by the specs and working on that tiny little widget to make the UI better.
% 
% 3. Outlook
% 1. Introspective capacity is vital for robotics. I'm looking into it in the context of deep learning and I'd like to make it work, but it's not sth I want to do a PhD on; Masters might be good
% 2. Reinforcement Learning fascinates me in that it is, together with the inverse problem, more or less how we learn. I'd like to study it further for I believe it is THE way of bringing ML to the masses. It's way easier to specify a reward  function or to find an expert than to gather supervised training datasets. I'm interested into human learning processes as well and I wonder if we can port them to the neural network settings
% 3. Recurrent neural networks seem suitable to a plethora of various tasks. E.g. object classification, and even more object detection, seem to be doing a lot of redundant work when dealing with e.g. video sequences. I'd like to work on itroducing knowledge of what might be in the sequence as priors based on what we've already seen there. Another thing is to use RNNs to compute optical flow from a sequence, but using one image at a time. Also RNNs for optical flow from DVS would be cool. 
