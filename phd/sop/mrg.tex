\documentclass[12pt]{article}
\usepackage[utf8]{inputenc}
\setlength{\oddsidemargin}{0in}
\setlength{\evensidemargin}{0in}
\setlength{\textwidth}{6.5in}
\setlength{\topmargin}{-.3in}
\setlength{\textheight}{9in}
\pagestyle{empty}

\setlength{\parindent}{0em}
\setlength{\parskip}{1em}

\begin{document}

\begin{center}
{\Large Research Proposal} \\[.1in]
{\large Adam Kosiorek}
\end{center}

\vspace*{.1in}

Artificial Intelligence has always tantalized me. Long before understanding what computer science stands for I wondered why AI has not been achieved yet. To explore it further, I enrolled in BSc in Robotics at the best technical university in my country. Programming and software-oriented courses were my favourite and paired with my passion for vision and perception, which I have developed as a photojournalist, they steered me towards my first computer vision course. At that moment it became apparent that pursuing a PhD was one of the few ways that could get me closer to solving the AI challange.

To learn from the best, I joined Computer Vision Lab at Samsung R\&D, where I implemented a Bag of Words-based pipeline for object classification. Consequently, I learned about the state-of-the-art in keypoint detection and feature description, which later helped me to appreciate the importance of learned low level features in convolutional neural networks. Before jumping into deep learning for object classification and image duplicate detection, I ported the project to Android and researched codebook generation within BoW. In my bachelor thesis I investigated how spatial information affects classification performance by implementing a similar system for Kinect-gathered point cloud classification.

My observation was that deep learning, although often superior to other approaches, requires huge computational resources and datasets. To mitigate this problem I went to study computational science at the Technical University of Munich. Numerical linear algebra and scientific computing, comprising the majority of the coursework, help enormously with optimization and implementation of algorithms. Currently, I am working with Rudolph Triebel of the Computer Vision group on the introspective capacity of neural networks. It is understood as the ability of a classifier to assess uncertainty of its prediction given the data \cite{introspective}. It is a paramount problem in mobile robotics and medicine where wrong classification might lead to loss of life. Our intermediate results show that neural networks, augmented with additional layers and a novel cost function, can be jointly trained for classification and uncertainty estimation. The topic might expand into my master thesis.

Last summer I did an internship at Bloomberg in London, where I worked on fraud detection in financial transactions. The problem can be cast as unsupervised anomaly detection with further verification in a supervised setting. I learned how difficult it can be to introduce an innovative approach in a corporate environment.

In my doctoral research project I would like to focus on deep neural networks for reinforcement learning and recurrent neural networks. 

My passion for solving technical problems coupled with my demonstrated skills show that I am ready and extremely motivated to carry out research as a PhD student at the University of Oxford. I am confident that I will be able to contribute significantly to and benefit immensely from my stay in Oxford. Thank you for considering my application.

\bibliographystyle{ieeetr}
\bibliography{bibliography}
\end{document}