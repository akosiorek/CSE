\documentclass[mathserif, 10pt]{beamer}
\usepackage{bm}
\usepackage{xspace}
\usepackage{amssymb}
\usepackage{amsmath}
\usepackage{amsfonts}
\usepackage{euscript}

\newcommand{\bba}{{\bf a}}
\newcommand{\bbb}{{\bf b}}
\newcommand{\bbc}{{\bf c}}
\newcommand{\bbd}{{\bf d}}
\newcommand{\bbe}{{\bf e}}
\newcommand{\bbf}{{\bf f}}
\newcommand{\bbg}{{\bf g}}
\newcommand{\bbh}{{\bf h}}
\newcommand{\bbi}{{\bf i}}
\newcommand{\bbj}{{\bf j}}
\newcommand{\bbk}{{\bf k}}
\newcommand{\bbl}{{\bf l}}
\newcommand{\bbm}{{\bf m}}
\newcommand{\bbn}{{\bf n}}
\newcommand{\bbo}{{\bf o}}
\newcommand{\bbp}{{\bf p}}
\newcommand{\bbq}{{\bf q}}
\newcommand{\bbr}{{\bf r}}
\newcommand{\bbs}{{\bf s}}
\newcommand{\bbt}{{\bf t}}
\newcommand{\bbu}{{\bf u}}
\newcommand{\bbv}{{\bf v}}
\newcommand{\bbx}{{\bf x}}
\newcommand{\bby}{{\bf y}}
\newcommand{\bbz}{{\bf z}}
\newcommand{\bbw}{{\bf w}}

\newcommand{\bbA}{{\bf A}}
\newcommand{\bbB}{{\bf B}}
\newcommand{\bbC}{{\bf C}}
\newcommand{\bbD}{{\bf D}}
\newcommand{\bbE}{{\bf E}}
\newcommand{\bbF}{{\bf F}}
\newcommand{\bbG}{{\bf G}}
\newcommand{\bbH}{{\bf H}}
\newcommand{\bbI}{{\bf I}}
\newcommand{\bbJ}{{\bf J}}
\newcommand{\bbK}{{\bf K}}
\newcommand{\bbL}{{\bf L}}
\newcommand{\bbM}{{\bf M}}
\newcommand{\bbN}{{\bf N}}
\newcommand{\bbO}{{\bf O}}
\newcommand{\bbP}{{\bf P}}
\newcommand{\bbQ}{{\bf Q}}
\newcommand{\bbR}{{\bf R}}
\newcommand{\bbS}{{\bf S}}
\newcommand{\bbT}{{\bf T}}
\newcommand{\bbU}{{\bf U}}
\newcommand{\bbV}{{\bf V}}
\newcommand{\bbX}{{\bf X}}
\newcommand{\bbZ}{{\bf Z}}
\newcommand{\bbY}{{\bf Y}}
\newcommand{\bbW}{{\bf W}}

\newcommand{\cca}{{\cal a}}
\newcommand{\ccb}{{\cal b}}
\newcommand{\ccc}{{\cal c}}
\newcommand{\ccd}{{\cal d}}
\newcommand{\cce}{{\cal e}}
\newcommand{\ccf}{{\cal f}}
\newcommand{\ccg}{{\cal g}}
\newcommand{\cch}{{\cal h}}
\newcommand{\cci}{{\cal i}}
\newcommand{\ccj}{{\cal j}}
\newcommand{\cck}{{\cal k}}
\newcommand{\ccl}{{\cal l}}
\newcommand{\ccm}{{\cal m}}
\newcommand{\ccn}{{\cal n}}
\newcommand{\cco}{{\cal o}}
\newcommand{\ccp}{{\cal p}}
\newcommand{\ccq}{{\cal q}}
\newcommand{\ccr}{{\cal r}}
\newcommand{\ccs}{{\cal s}}
\newcommand{\cct}{{\cal t}}
\newcommand{\ccu}{{\cal u}}
\newcommand{\ccv}{{\cal v}}
\newcommand{\ccx}{{\cal x}}
\newcommand{\ccy}{{\cal y}}
\newcommand{\ccz}{{\cal z}}
\newcommand{\ccw}{{\cal w}}

\newcommand{\ccA}{{\cal A}}
\newcommand{\ccB}{{\cal B}}
\newcommand{\ccC}{{\cal C}}
\newcommand{\ccD}{{\cal D}}
\newcommand{\ccE}{{\cal E}}
\newcommand{\ccF}{{\cal F}}
\newcommand{\ccG}{{\cal G}}
\newcommand{\ccH}{{\cal H}}
\newcommand{\ccI}{{\cal I}}
\newcommand{\ccJ}{{\cal J}}
\newcommand{\ccK}{{\cal K}}
\newcommand{\ccL}{{\cal L}}
\newcommand{\ccM}{{\cal M}}
\newcommand{\ccN}{{\cal N}}
\newcommand{\ccO}{{\cal O}}
\newcommand{\ccP}{{\cal P}}
\newcommand{\ccQ}{{\cal Q}}
\newcommand{\ccR}{{\cal R}}
\newcommand{\ccS}{{\cal S}}
\newcommand{\ccT}{{\cal T}}
\newcommand{\ccU}{{\cal U}}
\newcommand{\ccV}{{\cal V}}
\newcommand{\ccX}{{\cal X}}
\newcommand{\ccZ}{{\cal Z}}
\newcommand{\ccY}{{\cal Y}}
\newcommand{\ccW}{{\cal W}}

\newcommand{\bbwth}{\widetilde{\bf{h}}}
\newcommand{\bbwtx}{\widetilde{\bf{x}}}
\newcommand{\bbwtA}{\widetilde{\bf{A}}}
\newcommand{\bbwtH}{\widetilde{\bf{H}}}
\newcommand{\bbwtU}{\widetilde{\bf{U}}}
\newcommand{\bbwtS}{\widetilde{\bf{S}}}
\newcommand{\bbwtV}{\widetilde{\bf{V}}}

\newcommand{\BJ}{{\bf J}}
\newcommand{\mP}{{\mathcal P}}
\newcommand{\mU}{{\mathcal U}}
\newcommand{\mV}{{\mathcal V}}
\newcommand{\mX}{{\mathcal X}}
\newcommand{\rank}{{\mbox{rank}}}
\newcommand{\diag}{{\mbox{diag}}}
\newcommand{\kernel}{{\mbox{Ker}}}
\newcommand{\image}{{\mbox{Img}}}

\newcommand{\inv}{{-1}}

\newcommand{\bgamma}{\hbox{\boldmath $\gamma$}}
\newcommand{\bGamma}{\hbox{\boldmath $\Gamma$}}

\newcommand{\bnabla}{\hbox{\boldmath $\nabla$}}
%\newcommand{\bNabla}{\hbox{\boldmath $\Nabla$}}

\newcommand{\bups}{\hbox{\boldmath $\upsilon$}}
\newcommand{\bUps}{\hbox{\boldmath $\Upsilon$}}

\newcommand{\bomega}{\hbox{\boldmath $\omega$}}
\newcommand{\bOmega}{\hbox{\boldmath $\Omega$}}

\newcommand{\bpi}{\hbox{\boldmath $\pi$}}
\newcommand{\bPi}{\hbox{\boldmath $\Pi$}}

\newcommand{\bnu}{\hbox{\boldmath $\nu$}}
\newcommand{\bmu}{\hbox{\boldmath $\mu$}}

\newcommand{\sinc}{\mbox{sinc}}
\newcommand{\mbbZ}{{\mathbb Z}}
\newcommand{\mbbR}{{\mathbb R}}
\newcommand{\mbbP}{{\mathbb P}}

\newcommand{\bcX}{\hbox{\boldmath $\cal X$}}

\newcommand{\bzeta}{\hbox{\boldmath $\zeta$}}

\newcommand{\bxi}{\hbox{\boldmath $\xi$}}
\newcommand{\bchi}{\hbox{\boldmath $\chi$}}

\newcommand{\brho}{\hbox{\boldmath $\rho$}}

\newcommand{\blambda}{\hbox{\boldmath $\lambda$}}

\newcommand{\bveps}{\hbox{\boldmath $\varepsilon$}}

\newcommand{\bphi}{\hbox{\boldmath $\phi$}}
\newcommand{\bPhi}{\hbox{\boldmath $\Phi$}}

\newcommand{\bdelta}{\hbox{\boldmath $\delta$}}
\newcommand{\bDelta}{\hbox{\boldmath $\Delta$}}

\newcommand{\bPsi}{\hbox{\boldmath $\Psi$}}


\newcommand{\bvphi}{\hbox{\boldmath $\varphi$}}
\newcommand{\btau}{\hbox{\boldmath $\tau$}}
\newcommand{\bsigma}{\hbox{\boldmath $\sigma$}}
\newcommand{\bbeta}{\hbox{\boldmath $\beta$}}

%\renewcommand{\subfigcapskip}{1pt}
%\renewcommand{\subfigbottomskip}{1pt}

\newtheorem{Proposition}{\bf Proposition}
%\newtheorem{Theorem}{Theorem}
%\newtheorem{lemma}{Lemma}
%\newtheorem{theorproof}{Proof}

\newtheorem{propopreuve}{{\bf Preuve de la proposition}}
%\theoremstyle{break} 
\newtheorem{theoreme}{{\bf Th�or�me}}
\newtheorem{theorpreuve}{{\bf Preuve du th�or�me}}
\newtheorem{lemme}{{\bf Lemme}}
\newtheorem{lemmepreuve}{{\bf Preuve du lemme}}

\def\QED{\mbox{\rule[0pt]{1.5ex}{1.5ex}}}
\def\proof{\noindent\hspace{2em}{\it Proof: }}
%\def\endproof{\hspace*{\fill}~\QED\par\endtrivlist\unskip}
\def\endproof{\par}



\newcommand{\wha}{{\widehat a}}
\newcommand{\whb}{{\widehat b}}
\newcommand{\whc}{{\widehat c}}
\newcommand{\whd}{{\widehat d}}
\newcommand{\whe}{{\widehat e}}
\newcommand{\whf}{{\widehat f}}
\newcommand{\whg}{{\widehat g}}
\newcommand{\whh}{{\widehat h}}
\newcommand{\whi}{{\widehat i}}
\newcommand{\whj}{{\widehat j}}
\newcommand{\whk}{{\widehat k}}
\newcommand{\whl}{{\widehat l}}
\newcommand{\whm}{{\widehat m}}
\newcommand{\whn}{{\widehat n}}
\newcommand{\who}{{\widehat o}}
\newcommand{\whp}{{\widehat p}}
\newcommand{\whq}{{\widehat q}}
\newcommand{\whr}{{\widehat r}}
\newcommand{\whs}{{\widehat s}}
\newcommand{\wht}{{\widehat t}}
\newcommand{\whu}{{\widehat u}}
\newcommand{\whv}{{\widehat v}}
\newcommand{\whx}{{\widehat x}}
\newcommand{\why}{{\widehat y}}
\newcommand{\whz}{{\widehat z}}
\newcommand{\whw}{{\widehat w}}

\newcommand{\whA}{{\widehat A}}
\newcommand{\whB}{{\widehat B}}
\newcommand{\whC}{{\widehat C}}
\newcommand{\whD}{{\widehat D}}
\newcommand{\whE}{{\widehat E}}
\newcommand{\whF}{{\widehat F}}
\newcommand{\whG}{{\widehat G}}
\newcommand{\whH}{{\widehat H}}
\newcommand{\whI}{{\widehat I}}
\newcommand{\whJ}{{\widehat J}}
\newcommand{\whK}{{\widehat K}}
\newcommand{\whL}{{\widehat L}}
\newcommand{\whM}{{\widehat M}}
\newcommand{\whN}{{\widehat N}}
\newcommand{\whO}{{\widehat O}}
\newcommand{\whP}{{\widehat P}}
\newcommand{\whQ}{{\widehat Q}}
\newcommand{\whR}{{\widehat R}}
\newcommand{\whS}{{\widehat S}}
\newcommand{\whT}{{\widehat T}}
\newcommand{\whU}{{\widehat U}}
\newcommand{\whV}{{\widehat V}}
\newcommand{\whZ}{{\widehat X}}
\newcommand{\whY}{{\widehat Y}}
\newcommand{\whW}{{\widehat W}}

\newcommand{\whba}{{\widehat {\bf a}}}
\newcommand{\whbb}{{\widehat {\bf b}}}
\newcommand{\whbc}{{\widehat {\bf c}}}
\newcommand{\whbd}{{\widehat {\bf d}}}
\newcommand{\whbe}{{\widehat {\bf e}}}
\newcommand{\whbf}{{\widehat {\bf f}}}
\newcommand{\whbg}{{\widehat {\bf g}}}
\newcommand{\whbh}{{\widehat {\bf h}}}
\newcommand{\whbi}{{\widehat {\bf i}}}
\newcommand{\whbj}{{\widehat {\bf j}}}
\newcommand{\whbk}{{\widehat {\bf k}}}
\newcommand{\whbl}{{\widehat {\bf l}}}
\newcommand{\whbm}{{\widehat {\bf m}}}
\newcommand{\whbn}{{\widehat {\bf n}}}
\newcommand{\whbo}{{\widehat {\bf o}}}
\newcommand{\whbp}{{\widehat {\bf p}}}
\newcommand{\whbq}{{\widehat {\bf q}}}
\newcommand{\whbr}{{\widehat {\bf r}}}
\newcommand{\whbs}{{\widehat {\bf s}}}
\newcommand{\whbt}{{\widehat {\bf t}}}
\newcommand{\whbu}{{\widehat {\bf u}}}
\newcommand{\whbv}{{\widehat {\bf v}}}
\newcommand{\whbx}{{\widehat {\bf x}}}
\newcommand{\whby}{{\widehat {\bf y}}}
\newcommand{\whbz}{{\widehat {\bf z}}}
\newcommand{\whbw}{{\widehat {\bf w}}}

\newcommand{\whbA}{{\widehat {\bf A}}}
\newcommand{\whbB}{{\widehat {\bf B}}}
\newcommand{\whbC}{{\widehat {\bf C}}}
\newcommand{\whbD}{{\widehat {\bf D}}}
\newcommand{\whbE}{{\widehat {\bf E}}}
\newcommand{\whbF}{{\widehat {\bf F}}}
\newcommand{\whbG}{{\widehat {\bf G}}}
\newcommand{\whbH}{{\widehat {\bf H}}}
\newcommand{\whbI}{{\widehat {\bf I}}}
\newcommand{\whbJ}{{\widehat {\bf J}}}
\newcommand{\whbK}{{\widehat {\bf K}}}
\newcommand{\whbL}{{\widehat {\bf L}}}
\newcommand{\whbM}{{\widehat {\bf M}}}
\newcommand{\whbN}{{\widehat {\bf N}}}
\newcommand{\whbO}{{\widehat {\bf O}}}
\newcommand{\whbP}{{\widehat {\bf P}}}
\newcommand{\whbQ}{{\widehat {\bf Q}}}
\newcommand{\whbR}{{\widehat {\bf R}}}
\newcommand{\whbS}{{\widehat {\bf S}}}
\newcommand{\whbT}{{\widehat {\bf T}}}
\newcommand{\whbU}{{\widehat {\bf U}}}
\newcommand{\whbV}{{\widehat {\bf V}}}
\newcommand{\whbZ}{{\widehat {\bf X}}}
\newcommand{\whbY}{{\widehat {\bf Y}}}
\newcommand{\whbW}{{\widehat {\bf W}}}

\newcommand{\wta}{{\widetilde a}}
\newcommand{\wtb}{{\widetilde b}}
\newcommand{\wtc}{{\widetilde c}}
\newcommand{\wtd}{{\widetilde d}}
\newcommand{\wte}{{\widetilde e}}
\newcommand{\wtf}{{\widetilde f}}
\newcommand{\wtg}{{\widetilde g}}
\newcommand{\wth}{{\widetilde h}}
\newcommand{\wti}{{\widetilde i}}
\newcommand{\wtj}{{\widetilde j}}
\newcommand{\wtk}{{\widetilde k}}
\newcommand{\wtl}{{\widetilde l}}
\newcommand{\wtm}{{\widetilde m}}
\newcommand{\wtn}{{\widetilde n}}
\newcommand{\wto}{{\widetilde o}}
\newcommand{\wtp}{{\widetilde p}}
\newcommand{\wtq}{{\widetilde q}}
\newcommand{\wtr}{{\widetilde r}}
\newcommand{\wts}{{\widetilde s}}
\newcommand{\wtt}{{\widetilde t}}
\newcommand{\wtu}{{\widetilde u}}
\newcommand{\wtv}{{\widetilde v}}
\newcommand{\wtx}{{\widetilde x}}
\newcommand{\wty}{{\widetilde y}}
\newcommand{\wtz}{{\widetilde z}}
\newcommand{\wtw}{{\widetilde w}}

\newcommand{\wtA}{{\widetilde A}}
\newcommand{\wtB}{{\widetilde B}}
\newcommand{\wtC}{{\widetilde C}}
\newcommand{\wtD}{{\widetilde D}}
\newcommand{\wtE}{{\widetilde E}}
\newcommand{\wtF}{{\widetilde F}}
\newcommand{\wtG}{{\widetilde G}}
\newcommand{\wtH}{{\widetilde H}}
\newcommand{\wtI}{{\widetilde I}}
\newcommand{\wtJ}{{\widetilde J}}
\newcommand{\wtK}{{\widetilde K}}
\newcommand{\wtL}{{\widetilde L}}
\newcommand{\wtM}{{\widetilde M}}
\newcommand{\wtN}{{\widetilde N}}
\newcommand{\wtO}{{\widetilde O}}
\newcommand{\wtP}{{\widetilde P}}
\newcommand{\wtQ}{{\widetilde Q}}
\newcommand{\wtR}{{\widetilde R}}
\newcommand{\wtS}{{\widetilde S}}
\newcommand{\wtT}{{\widetilde T}}
\newcommand{\wtU}{{\widetilde U}}
\newcommand{\wtV}{{\widetilde V}}
\newcommand{\wtZ}{{\widetilde X}}
\newcommand{\wtY}{{\widetilde Y}}
\newcommand{\wtW}{{\widetilde W}}

\newcommand{\wtba}{{\widetilde {\bf a}}}
\newcommand{\wtbb}{{\widetilde {\bf b}}}
\newcommand{\wtbc}{{\widetilde {\bf c}}}
\newcommand{\wtbd}{{\widetilde {\bf d}}}
\newcommand{\wtbe}{{\widetilde {\bf e}}}
\newcommand{\wtbf}{{\widetilde {\bf f}}}
\newcommand{\wtbg}{{\widetilde {\bf g}}}
\newcommand{\wtbh}{{\widetilde {\bf h}}}
\newcommand{\wtbi}{{\widetilde {\bf i}}}
\newcommand{\wtbj}{{\widetilde {\bf j}}}
\newcommand{\wtbk}{{\widetilde {\bf k}}}
\newcommand{\wtbl}{{\widetilde {\bf l}}}
\newcommand{\wtbm}{{\widetilde {\bf m}}}
\newcommand{\wtbn}{{\widetilde {\bf n}}}
\newcommand{\wtbo}{{\widetilde {\bf o}}}
\newcommand{\wtbp}{{\widetilde {\bf p}}}
\newcommand{\wtbq}{{\widetilde {\bf q}}}
\newcommand{\wtbr}{{\widetilde {\bf r}}}
\newcommand{\wtbs}{{\widetilde {\bf s}}}
\newcommand{\wtbt}{{\widetilde {\bf t}}}
\newcommand{\wtbu}{{\widetilde {\bf u}}}
\newcommand{\wtbv}{{\widetilde {\bf v}}}
\newcommand{\wtbx}{{\widetilde {\bf x}}}
\newcommand{\wtby}{{\widetilde {\bf y}}}
\newcommand{\wtbz}{{\widetilde {\bf z}}}
\newcommand{\wtbw}{{\widetilde {\bf w}}}

\newcommand{\wtbA}{{\widetilde {\bf A}}}
\newcommand{\wtbB}{{\widetilde {\bf B}}}
\newcommand{\wtbC}{{\widetilde {\bf C}}}
\newcommand{\wtbD}{{\widetilde {\bf D}}}
\newcommand{\wtbE}{{\widetilde {\bf E}}}
\newcommand{\wtbF}{{\widetilde {\bf F}}}
\newcommand{\wtbG}{{\widetilde {\bf G}}}
\newcommand{\wtbH}{{\widetilde {\bf H}}}
\newcommand{\wtbI}{{\widetilde {\bf I}}}
\newcommand{\wtbJ}{{\widetilde {\bf J}}}
\newcommand{\wtbK}{{\widetilde {\bf K}}}
\newcommand{\wtbL}{{\widetilde {\bf L}}}
\newcommand{\wtbM}{{\widetilde {\bf M}}}
\newcommand{\wtbN}{{\widetilde {\bf N}}}
\newcommand{\wtbO}{{\widetilde {\bf O}}}
\newcommand{\wtbP}{{\widetilde {\bf P}}}
\newcommand{\wtbQ}{{\widetilde {\bf Q}}}
\newcommand{\wtbR}{{\widetilde {\bf R}}}
\newcommand{\wtbS}{{\widetilde {\bf S}}}
\newcommand{\wtbT}{{\widetilde {\bf T}}}
\newcommand{\wtbU}{{\widetilde {\bf U}}}
\newcommand{\wtbV}{{\widetilde {\bf V}}}
\newcommand{\wtbX}{{\widetilde {\bf X}}}
\newcommand{\wtbY}{{\widetilde {\bf Y}}}
\newcommand{\wtbW}{{\widetilde {\bf W}}}
\newcommand{\wtbcX}{{\widetilde {\boldmath $\cal X$}}}
\newcommand{\hx}{\hat{x}}
\newcommand{\hy}{\hat{y}}
\newcommand{\hw}{\hat{w}}


\def\olba {\ensuremath{{\overline {\bf a}}\xspace}}
\def\olbb {\ensuremath{{\overline {\bf b}}\xspace}}
\def\olbc {\ensuremath{{\overline {\bf c}}\xspace}}
\def\olbd {\ensuremath{{\overline {\bf d}}\xspace}}
\def\olbe {\ensuremath{{\overline {\bf e}}\xspace}}
\def\olbf {\ensuremath{{\overline {\bf f}}\xspace}}
\def\olbg {\ensuremath{{\overline {\bf g}}\xspace}}
\def\olbh {\ensuremath{{\overline {\bf h}}\xspace}}
\def\olbi {\ensuremath{{\overline {\bf i}}\xspace}}
\def\olbj {\ensuremath{{\overline {\bf j}}\xspace}}
\def\olbk {\ensuremath{{\overline {\bf k}}\xspace}}
\def\olbl {\ensuremath{{\overline {\bf l}}\xspace}}
\def\olbm {\ensuremath{{\overline {\bf m}}\xspace}}
\def\olbn {\ensuremath{{\overline {\bf n}}\xspace}}
\def\olbo {\ensuremath{{\overline {\bf o}}\xspace}}
\def\olbp {\ensuremath{{\overline {\bf p}}\xspace}}
\def\olbq {\ensuremath{{\overline {\bf q}}\xspace}}
\def\olbr {\ensuremath{{\overline {\bf r}}\xspace}}
\def\olbs {\ensuremath{{\overline {\bf s}}\xspace}}
\def\olbt {\ensuremath{{\overline {\bf t}}\xspace}}
\def\olbu {\ensuremath{{\overline {\bf u}}\xspace}}
\def\olbv {\ensuremath{{\overline {\bf v}}\xspace}}
\def\olbx {\ensuremath{{\overline {\bf x}}\xspace}}
\def\olby {\ensuremath{{\overline {\bf y}}\xspace}}
\def\olbz {\ensuremath{{\overline {\bf z}}\xspace}}
\def\olbw {\ensuremath{{\overline {\bf w}}\xspace}}

\def\olbA {\ensuremath{{\overline {\bf A}}\xspace}}
\def\olbB {\ensuremath{{\overline {\bf B}}\xspace}}
\def\olbC {\ensuremath{{\overline {\bf C}}\xspace}}
\def\olbD {\ensuremath{{\overline {\bf D}}\xspace}}
\def\olbE {\ensuremath{{\overline {\bf E}}\xspace}}
\def\olbF {\ensuremath{{\overline {\bf F}}\xspace}}
\def\olbG {\ensuremath{{\overline {\bf G}}\xspace}}
\def\olbH {\ensuremath{{\overline {\bf H}}\xspace}}
\def\olbI {\ensuremath{{\overline {\bf I}}\xspace}}
\def\olbJ {\ensuremath{{\overline {\bf J}}\xspace}}
\def\olbK {\ensuremath{{\overline {\bf K}}\xspace}}
\def\olbL {\ensuremath{{\overline {\bf L}}\xspace}}
\def\olbM {\ensuremath{{\overline {\bf M}}\xspace}}
\def\olbN {\ensuremath{{\overline {\bf N}}\xspace}}
\def\olbO {\ensuremath{{\overline {\bf O}}\xspace}}
\def\olbP {\ensuremath{{\overline {\bf P}}\xspace}}
\def\olbQ {\ensuremath{{\overline {\bf Q}}\xspace}}
\def\olbR {\ensuremath{{\overline {\bf R}}\xspace}}
\def\olbS {\ensuremath{{\overline {\bf S}}\xspace}}
\def\olbT {\ensuremath{{\overline {\bf T}}\xspace}}
\def\olbU {\ensuremath{{\overline {\bf U}}\xspace}}
\def\olbV {\ensuremath{{\overline {\bf V}}\xspace}}
\def\olbZ {\ensuremath{{\overline {\bf X}}\xspace}}
\def\olbY {\ensuremath{{\overline {\bf Y}}\xspace}}
\def\olbW {\ensuremath{{\overline {\bf W}}\xspace}}


\def\ola {\ensuremath{{\overline {a}}\xspace}}
\def\olb {\ensuremath{{\overline {b}}\xspace}}
\def\olc {\ensuremath{{\overline {c}}\xspace}}
\def\old {\ensuremath{{\overline {d}}\xspace}}
\def\ole {\ensuremath{{\overline {e}}\xspace}}
\def\olf {\ensuremath{{\overline {f}}\xspace}}
\def\olg {\ensuremath{{\overline {g}}\xspace}}
\def\olh {\ensuremath{{\overline {h}}\xspace}}
\def\oli {\ensuremath{{\overline {i}}\xspace}}
\def\olj {\ensuremath{{\overline {j}}\xspace}}
\def\olk {\ensuremath{{\overline {k}}\xspace}}
\def\oll {\ensuremath{{\overline {l}}\xspace}}
\def\olm {\ensuremath{{\overline {m}}\xspace}}
\def\oln {\ensuremath{{\overline {n}}\xspace}}
\def\olo {\ensuremath{{\overline {o}}\xspace}}
\def\olp {\ensuremath{{\overline {p}}\xspace}}
\def\olq {\ensuremath{{\overline {q}}\xspace}}
\def\olr {\ensuremath{{\overline {r}}\xspace}}
\def\ols {\ensuremath{{\overline {s}}\xspace}}
\def\olt {\ensuremath{{\overline {t}}\xspace}}
\def\olu {\ensuremath{{\overline {u}}\xspace}}
\def\olv {\ensuremath{{\overline {v}}\xspace}}
\def\olw {\ensuremath{{\overline {w}}\xspace}}
\def\olx {\ensuremath{{\overline {x}}\xspace}}
\def\oly {\ensuremath{{\overline {y}}\xspace}}
\def\olz {\ensuremath{{\overline {z}}\xspace}}


\def\wta {\ensuremath{{\widetilde a}\xspace}}
\def\wtb {\ensuremath{{\widetilde b}\xspace}}
\def\wtc {\ensuremath{{\widetilde c}\xspace}}
\def\wtd {\ensuremath{{\widetilde d}\xspace}}
\def\wte {\ensuremath{{\widetilde e}\xspace}}
\def\wtf {\ensuremath{{\widetilde f}\xspace}}
\def\wtg {\ensuremath{{\widetilde g}\xspace}}
\def\wth {\ensuremath{{\widetilde h}\xspace}}
\def\wti {\ensuremath{{\widetilde i}\xspace}}
\def\wtj {\ensuremath{{\widetilde j}\xspace}}
\def\wtk {\ensuremath{{\widetilde k}\xspace}}
\def\wtl {\ensuremath{{\widetilde l}\xspace}}
\def\wtm {\ensuremath{{\widetilde m}\xspace}}
\def\wtn {\ensuremath{{\widetilde n}\xspace}}
\def\wto {\ensuremath{{\widetilde o}\xspace}}
\def\wtp {\ensuremath{{\widetilde p}\xspace}}
\def\wtq {\ensuremath{{\widetilde q}\xspace}}
\def\wtr {\ensuremath{{\widetilde r}\xspace}}
\def\wts {\ensuremath{{\widetilde s}\xspace}}
\def\wtt {\ensuremath{{\widetilde t}\xspace}}
\def\wtu {\ensuremath{{\widetilde u}\xspace}}
\def\wtv {\ensuremath{{\widetilde v}\xspace}}
\def\wtx {\ensuremath{{\widetilde x}\xspace}}
\def\wty {\ensuremath{{\widetilde y}\xspace}}
\def\wtz {\ensuremath{{\widetilde z}\xspace}}
\def\wtw {\ensuremath{{\widetilde w}\xspace}}

\def\wtA {\ensuremath{{\widetilde A}\xspace}}
\def\wtB {\ensuremath{{\widetilde B}\xspace}}
\def\wtC {\ensuremath{{\widetilde C}\xspace}}
\def\wtD {\ensuremath{{\widetilde D}\xspace}}
\def\wtE {\ensuremath{{\widetilde E}\xspace}}
\def\wtF {\ensuremath{{\widetilde F}\xspace}}
\def\wtG {\ensuremath{{\widetilde G}\xspace}}
\def\wtH {\ensuremath{{\widetilde H}\xspace}}
\def\wtI {\ensuremath{{\widetilde I}\xspace}}
\def\wtJ {\ensuremath{{\widetilde J}\xspace}}
\def\wtK {\ensuremath{{\widetilde K}\xspace}}
\def\wtL {\ensuremath{{\widetilde L}\xspace}}
\def\wtM {\ensuremath{{\widetilde M}\xspace}}
\def\wtN {\ensuremath{{\widetilde N}\xspace}}
\def\wtO {\ensuremath{{\widetilde O}\xspace}}
\def\wtP {\ensuremath{{\widetilde P}\xspace}}
\def\wtQ {\ensuremath{{\widetilde Q}\xspace}}
\def\wtR {\ensuremath{{\widetilde R}\xspace}}
\def\wtS {\ensuremath{{\widetilde S}\xspace}}
\def\wtT {\ensuremath{{\widetilde T}\xspace}}
\def\wtU {\ensuremath{{\widetilde U}\xspace}}
\def\wtV {\ensuremath{{\widetilde V}\xspace}}
\def\wtZ {\ensuremath{{\widetilde X}\xspace}}
\def\wtY {\ensuremath{{\widetilde Y}\xspace}}
\def\wtW {\ensuremath{{\widetilde W}\xspace}}

\def\wtba {\ensuremath{{\widetilde {\bf a}}\xspace}}
\def\wtbb {\ensuremath{{\widetilde {\bf b}}\xspace}}
\def\wtbc {\ensuremath{{\widetilde {\bf c}}\xspace}}
\def\wtbd {\ensuremath{{\widetilde {\bf d}}\xspace}}
\def\wtbe {\ensuremath{{\widetilde {\bf e}}\xspace}}
\def\wtbf {\ensuremath{{\widetilde {\bf f}}\xspace}}
\def\wtbg {\ensuremath{{\widetilde {\bf g}}\xspace}}
\def\wtbh {\ensuremath{{\widetilde {\bf h}}\xspace}}
\def\wtbi {\ensuremath{{\widetilde {\bf i}}\xspace}}
\def\wtbj {\ensuremath{{\widetilde {\bf j}}\xspace}}
\def\wtbk {\ensuremath{{\widetilde {\bf k}}\xspace}}
\def\wtbl {\ensuremath{{\widetilde {\bf l}}\xspace}}
\def\wtbm {\ensuremath{{\widetilde {\bf m}}\xspace}}
\def\wtbn {\ensuremath{{\widetilde {\bf n}}\xspace}}
\def\wtbo {\ensuremath{{\widetilde {\bf o}}\xspace}}
\def\wtbp {\ensuremath{{\widetilde {\bf p}}\xspace}}
\def\wtbq {\ensuremath{{\widetilde {\bf q}}\xspace}}
\def\wtbr {\ensuremath{{\widetilde {\bf r}}\xspace}}
\def\wtbs {\ensuremath{{\widetilde {\bf s}}\xspace}}
\def\wtbt {\ensuremath{{\widetilde {\bf t}}\xspace}}
\def\wtbu {\ensuremath{{\widetilde {\bf u}}\xspace}}
\def\wtbv {\ensuremath{{\widetilde {\bf v}}\xspace}}
\def\wtbx {\ensuremath{{\widetilde {\bf x}}\xspace}}
\def\wtby {\ensuremath{{\widetilde {\bf y}}\xspace}}
\def\wtbz {\ensuremath{{\widetilde {\bf z}}\xspace}}
\def\wtbw {\ensuremath{{\widetilde {\bf w}}\xspace}}

\def\wtbA {\ensuremath{{\widetilde {\bf A}}\xspace}}
\def\wtbB {\ensuremath{{\widetilde {\bf B}}\xspace}}
\def\wtbC {\ensuremath{{\widetilde {\bf C}}\xspace}}
\def\wtbD {\ensuremath{{\widetilde {\bf D}}\xspace}}
\def\wtbE {\ensuremath{{\widetilde {\bf E}}\xspace}}
\def\wtbF {\ensuremath{{\widetilde {\bf F}}\xspace}}
\def\wtbG {\ensuremath{{\widetilde {\bf G}}\xspace}}
\def\wtbH {\ensuremath{{\widetilde {\bf H}}\xspace}}
\def\wtbI {\ensuremath{{\widetilde {\bf I}}\xspace}}
\def\wtbJ {\ensuremath{{\widetilde {\bf J}}\xspace}}
\def\wtbK {\ensuremath{{\widetilde {\bf K}}\xspace}}
\def\wtbL {\ensuremath{{\widetilde {\bf L}}\xspace}}
\def\wtbM {\ensuremath{{\widetilde {\bf M}}\xspace}}
\def\wtbN {\ensuremath{{\widetilde {\bf N}}\xspace}}
\def\wtbO {\ensuremath{{\widetilde {\bf O}}\xspace}}
\def\wtbP {\ensuremath{{\widetilde {\bf P}}\xspace}}
\def\wtbQ {\ensuremath{{\widetilde {\bf Q}}\xspace}}
\def\wtbR {\ensuremath{{\widetilde {\bf R}}\xspace}}
\def\wtbS {\ensuremath{{\widetilde {\bf S}}\xspace}}
\def\wtbT {\ensuremath{{\widetilde {\bf T}}\xspace}}
\def\wtbU {\ensuremath{{\widetilde {\bf U}}\xspace}}
\def\wtbV {\ensuremath{{\widetilde {\bf V}}\xspace}}
\def\wtbZ {\ensuremath{{\widetilde {\bf X}}\xspace}}
\def\wtbY {\ensuremath{{\widetilde {\bf Y}}\xspace}}
\def\wtbW {\ensuremath{{\widetilde {\bf W}}\xspace}}

\newcommand{\N}{\mbox{I\hspace{-.15em}N}}
%\newcommand{\R}{\mbox{I\hspace{-.15em}R}}
\newcommand{\C}{\mbox{l\hspace{-.47em}C}}
\newcommand{\G}{\mbox{l\hspace{-.47em}G}}
%\newcommand{\PP}{\mbox{I\hspace{-.15em}P}}

\def \smooth {{\cal C}^{\infty}}

\def \mbC {{\mathbb C}}
\def \mbE {{\mathbb E}}
\def \mbG {{\mathbb G}}
\def \mbL {{\mathbb L}}
\def \mbN {{\mathbb N}}
\def \mbO {{\mathbb O}}
\def \mbP {{\mathbb P}}
\def \mbQ {{\mathbb Q}}
\def \mbR {{\mathbb R}}
\def \mbS {{\mathbb S}}
\def \mbT {{\mathbb T}}
\def \mbZ {{\mathbb Z}} 

\def \mfg {{\mathfrak g}}
\def \mfs {{\mathfrak s}}
\def \mfl {{\mathfrak l}}

\def \got#1{\mathfrak{#1}}
\def \mbb#1{\mathbb{#1}}
\def\registered{{\ooalign {\hfil\raise .05ex\hbox{\scriptsize
R}\hfil\crcr\mathhexbox20D}}}


\def\REgistered{{\ooalign
{\hfil\raise.09ex\hbox{\tiny \sf R}\hfil\crcr\mathhexbox20D}}}

\newcommand{\norm}[2][\relax]{
\ifx#1\relax \ensuremath{\left\Vert#2\right\Vert}
\else \ensuremath{\left\Vert#2\right\Vert_{#1}}
\fi}

\newcommand{\D}[2]{\frac{\partial #1}{\partial #2}}



% \usepackage[font=footnotesize,labelfont=bf]{caption}
% \setbeamerfont{caption}{size=\scriptsize}

\mode<presentation>
{

  	\usetheme{Pittsburgh}

  	%\usetheme[numbers,
	 %pageofpages=of,% String used between the current page and the total page count.
         % bullet=circle,% Use circles instead of squares for bullets.
          %titleline=true,% Show a line below the frame title.
          %]{Singapore}
            
%  \setbeamertemplate{footline}[frame number]
%  \usefonttheme[onlysmall]{structurebold}
%  \setbeamercovered{dynamic}
% \setbeamercovered{transparent}

\usebackgroundtemplate{\includegraphics[height=\paperheight]{template}}

\setbeamertemplate{frametitle}{
	\vskip30pt
 	\usebeamercolor[blue!60!green]{frametitle}
  	\centering
  	\insertframetitle
	\vskip5pt
	\color{blue!60!green}{\hrule}
	}
}
\setbeamertemplate{footline}[frame number]
\setbeamercolor{math text}{fg=orange!80!black}
\newcommand{\hl}[1]{\textcolor{red!80!black}{\underline{#1}}}

\AtBeginSection[]{
  \begin{frame}
  \vfill
  \centering
  \begin{beamercolorbox}[sep=8pt,center,shadow=true,rounded=true]{title}
    \usebeamerfont{title}\insertsectionhead\par%
  \end{beamercolorbox}
  \vfill
  \end{frame}
}


\title[short title]{SemanticPaint}
\author[short presentator]{Adam Kosiorek}
\institute[CAMP]{Advisor: M.Eng.~Keisuke Tateno}
\date[]{10.12.2015}


\begin{document}

% 1. introduction
%   - lots of research in registration in visual understanding since the beginning of CV
%   - none approach so far achieved robust real time segmentation
%   - this paper introduces an end-to-end pipeline for interactive registration and segmentation of 3D enviornments
%   
% 2. State of the Art
%   - 
%   
% 3. Pipeline
%   - regristration and reconstruction
%   - classification
%   - crf and mean-field inference for smoothing
%   - VOPs
% 
%  4. Results
%   - dataset for segmentation and VOPs
%   - segmentation
%   - VOPs
%   - dataset for SRF
%   - SRF
%   
% 5. Discussion
%   - applications
%   - failures
%   
% 6. Summary
%   - what it is about
%   
% 7. Future Work
%   - scaling
%   - pure geometry
%   - (verbal) priors

%---------------------------------------------------------------------------------------

\begin{frame}
\titlepage
\end{frame}

%---------------------------------------------------------------------------------------

\begin{frame}{Outline}
\tableofcontents
\end{frame}


%---------------------------------------------------------------------------------------
\section{Introduction}

%---------------------------------------------------------------------------------------
\section{State of the Art}

\begin{frame}
\frametitle{Scene Understanding}

\begin{table}
\noindent\makebox[\linewidth]{
 \begin{tabular}{c|c|c}
  Who & What & How \\ \hline
  
  Valentin et. al. 2013  & inference on & RGB and geom. features  \\
  & mesh from TSDF  & CRF segmentation \\ \hline
    
  Kim et. al. 2013 & reconstruction & Voxel-based CRF \\ 
    & segmentation   & with visibility contraints\\ \hline
    
  Herbst et. al. 2014  & reconstruction & online model updates \\
     & segmentation & change detection\\ \hline
     

 \end{tabular}
}
\end{table}

\end{frame}


\begin{frame}
\frametitle{Model-based SLAM}

\begin{table}
\noindent\makebox[\linewidth]{
 \begin{tabular}{c|c|c}
  Who & What & How \\ \hline
  Newcombe et. al. 2011  & online 3D SLAM & model-based tracking \\
  && global TSDF volume \\ \hline
  
  Salas-Moreno et al. 2013 & object-level & offline object database \\
  & SLAM & pose-object graph\\ \hline
  
  Pradeep et.al. 2013 & 3D reconstruction & sparse tracking and  \\
    & with 1 RGB camera  & stereo reconstruction \\
    && on par with KinectFusion
 \end{tabular}
}
\end{table}


\end{frame}

%---------------------------------------------------------------------------------------
\section{Pipeline}

\begin{frame}
\frametitle{Pipeline Overview}
\begin{figure}
 \includegraphics[width=\textwidth]{figures/pipeline}
\end{figure}

\end{frame}

\begin{frame}
\frametitle{Voxel Oriented Patch features}

\begin{columns}
 \begin{column}{0.5\textwidth}
  \begin{figure}
  \center
  \includegraphics[width=\textwidth]{figures/vop}
  \caption{Colours shown in RGB for illustration purposes.}
  \label{fig:vop}
\end{figure}
 \end{column}
 \begin{column}{0.5\textwidth}
    $(\mathbf{p} - \mathbf{p}_i) \cdot \mathbf(n)_i = 0$ \\
    $r \times r$, $r = 13px$ with $10\frac{mm}{pixel}$\\
    CIELab\\
    Rotated to dominant gradient direction

  
 \end{column}

\end{columns}
\end{frame}

\begin{frame}
\frametitle{Random Forest}

\begin{columns}
 \begin{column}{0.5\textwidth}
  \begin{figure}
  \center
  \includegraphics[width=\textwidth]{figures/rf}
  \caption{Single tree}
\end{figure}
 \end{column}
 \begin{column}{0.5\textwidth}
    bagged trees\\
    greedy training\\
    bootstraped data\\    
    off-line, all data at once\\
    voting for final result\\    
    
    $(i, l) \in \mathcal{S}$ - (voxel, label) pairs\\
    $f(i, \theta)$ - split functions\\
    $\Theta$ - distribution of split functions\\
    $P_F(x_i = l | \mathbf{D})$ - class conditional probability\\
 \end{column}

\end{columns}
\end{frame}

\begin{frame}
\frametitle{Streaming Random Forest}
  \begin{itemize}
   \item Node n: Reservoir $R_n$ with a list of samples $T_n$, $|T_n| \leq K$
   \item First $K$ samples added
   \item Current samples swapped with new ones with decreasing probability
   \item Split node if: $|R_n| > N$
  \end{itemize}

  Information Gain:
  \vspace{-0.25cm}
  \begin{equation} \label{eq:infogain}
    G(R_n, R_n^L, R_n^R) = H(R_n) - \sum_{d \in \{L, R\}} \frac{|R_n^d|}{|R_n|}H(R_n^d)
  \end{equation}
  Shannon Entropy:
  \vspace{-0.25cm}
  \begin{equation}
   H(R_n) = - \sum_{(l, i) \in T_n} p(c_i = l) \log{p(c_i = l)}
  \end{equation}
  
  \vspace{-0.25cm}
  $H(R_n)$ computed from a node's class distribution
\end{frame}

\begin{frame}
\frametitle{SRF - Reservoir Splitting}
$m_n$ - number of samples seen at node $n$\\
$P(l | T_n)$ - normalized class distribution of $R_n$
\vspace{-0.25cm}
\begin{figure}[!ht]
  \center
  \includegraphics[width=\textwidth]{figures/forest}
%   \caption{Splitting reservoirs in Streaming Random Forest.}
  \label{fig:forest}
\end{figure}

\end{frame}

\begin{frame}
\frametitle{Dynamic Conditional Random Field}

Joint class probability distribution for the volume $\mathcal{V}$:
  \begin{equation} \label{eq:posterior}
  P(\mathbf{x}|\mathbf{D}) = \prod_{i \in \mathcal{V}} \left( \psi_i(x_i) \prod_{j \in \mathcal{E}_i} \psi_{ij}(x_i, x_j) \right) 
  \end{equation}

\vspace{-0.5cm}
Labeling Energy at time $t$:
  \begin{equation} \label{eq:energy}
  E_t(\mathbf{x}) = \sum_{i \in \mathcal{V}} \left( \phi_i(x_i) + \sum_{j \in \mathcal{E}_i} \phi_{ij} (x_i, x_j) \right) + K
  \end{equation}
  
\vspace{-0.5cm}  
where:\\
$\phi_i(x_i)$ - cost of assigning a label \\
$\phi_{ij}(x_i, x_j)$ - cost of assuming different labels\\
$\mathcal{E}_i$ - neighbourhood of voxel $i$

\end{frame}

\begin{frame}
\frametitle{CRF - User Interactions}
% \vspace{-1cm}
  Touching:
  \begin{columns}
   \begin{column}{0.25\textwidth}
    \begin{figure}
    \includegraphics[width=\textwidth]{figures/touch}
    \end{figure}
    \end{column}
    \begin{column}{0.75\textwidth}
      \begin{equation}
	\phi_i(l) =
	  \begin{cases}
	    0      & \quad \text{if } l = l_T\\
	    \infty  & \quad \text{otherwise}\\
	  \end{cases}
      \end{equation}
      \center
      $T$ --- touched pixels
   \end{column}

  \end{columns}
  
  \vspace{0.5cm}
  Encircling:
  \vspace{-0.5cm}
    \begin{columns}
   \begin{column}{0.25\textwidth}
    \begin{figure}
    \includegraphics[width=\textwidth]{figures/circle}
    \end{figure}
   \end{column}
    \begin{column}{0.75\textwidth}
      \begin{equation}
  \phi_i(l) =
    \begin{cases}
      \log P_E(fg|\mathbf{a}_i)       & \quad \text{if } l = \text{fg}\\
      \log (1 - P_E(fg|\mathbf{a}_i))  & \quad \text{if } l = \text{bg}\\
    \end{cases}
  \end{equation}
  
  \center
  $P_E$ from GMM\\
  fg --- inside\\
  bg --- outside
   \end{column}

  \end{columns}

\end{frame}

\begin{frame}
\frametitle{CRF - Predictions and Smoothnes}

  Predictions:
  \begin{equation}
  \phi_i(l) = -\log P_F(x_i = l | \mathbf{D})
  \end{equation}
  $P_F$ --- Streaming Random Forest prediction

  \vspace{0.5cm}
  Smoothnes:
  \begin{equation}
      \phi_{ij}(x_i, x_j)  = \theta_p e^{-||\mathbf{p}_i - \mathbf{p}_j||}  + \theta_a e^{-||\mathbf{a}_i - \mathbf{a}_j||} + \theta_n e^{-||\mathbf{n}_i - \mathbf{n}_j||}
  \end{equation}
  
  $\theta_p$, $\theta_a$, $\theta_n$ --- paramters \\
  $\mathbf{p}_i$ --- position\\
  $\mathbf{a}_i$ --- appearance\\
  $\mathbf{n}_i$ --- normal vector
  
\end{frame}

\begin{frame}
\frametitle{Mean-Field Inference}
$P(\mathbf{x})$ approximated by $Q(\mathbf{x})$ under $KL(Q||P)$:
\begin{equation}
 Q_i^t(l) = \frac{1}{Z_i}e^{M_i(l)} \text{, } t = 1, \ldots, T
\end{equation}
\begin{equation}
 M_i(l) = \phi_i(l) + \sum_{l' \in \mathcal{L}} \sum_{j \in \mathcal{E}_i} Q_j^{t-1}(l')\phi_{ij}(l, l')
\end{equation}


Frame at time $t$ initialized with:
\begin{equation}
 \widetilde{Q}_i^t(x_i) = \gamma Q_i^{t-1}(x_i) + (1 - \gamma) P_F^{t-1}(x_i = l | \mathbf{D}) \text{, } \gamma \in [0, 1]
\end{equation}

\end{frame}

%---------------------------------------------------------------------------------------
\section{Results}
\begin{frame}
\frametitle{Segmentation}
  \begin{figure}
   \includegraphics[width=\textwidth]{figures/results}
  \end{figure}

\end{frame}

\begin{frame}
\frametitle{Segmentation}

\begin{table}[!ht]
 \fontsize{10pt}{7.2}\selectfont
 \center
 \caption{Segmentation Results}
\noindent\makebox[\linewidth]{
  \begin{tabular}{c|c|c|c|c|c}
    \textbf{Component} & \textbf{LivingRoom} & \textbf{Bedroom} & \textbf{Kitchen} & \textbf{Desk} & \textbf{Average} \\ \hline
    User Interaction & 99.35\% & 97.61\% & 96.09\% & 97.73\% & 97.7\% \\ 
    Forest Prediction & 94.57\% & 88.31\% & 82.58\% & 90.29\% & 88.94\% \\
    Final Inference & 96.26\% & 95.19\% & 90.69\% & 95.55\% & 94.42\%
  \end{tabular}
 }
 \label{fig:segm_results}
\end{table}

\end{frame}

\begin{frame}
\frametitle{Features}
\vspace{-0.25cm}
 \begin{figure}[!ht]
  \center
  \includegraphics[width=\textwidth]{figures/results_vop}
%   \caption{Average Precision}
%   \label{fig:srf_ap}
  \end{figure}
\end{frame}

\begin{frame}
\frametitle{Features}

\begin{table}[!ht]
 \fontsize{10pt}{7.2}\selectfont
 \center
 \caption{Feature Comparison}
 \noindent\makebox[\linewidth]{
  \begin{tabular}{c|c|c|c|c|c}
  \textbf{Feature} & \textbf{LivingRoom} & \textbf{Bedroom} & \textbf{Kitchen} & \textbf{Desk} & \textbf{Average} \\ \hline
  VOP & \textbf{94.57\%} & \textbf{88.31\%} & 82.58\% & \textbf{90.29\%} & \textbf{88.94\%} \\
  $\Delta$ RGB mean & 80\% & 71.84\% & 76.29\% & 73.42\% & 75.39\% \\
  Depth Probe& 77.54\% & 61.79\% & \textbf{84.9\%} & 68.9\% & 73.06\% \\
  Color Probe& 56.39\% & 65.68\% & 60.77\% & 60.74\% & 60.9\% \\
  SURF & 43.74\% & 67.12\% & 57\% & 58.13\% & 56.5\% \\
  SPIN & 58.77\% & 43.22\% & 48.41\% & 36.1\% & 46.63\% \\
 \end{tabular}
 }
 \label{fig:vop_results}
\end{table}

\end{frame}

\begin{frame}
\frametitle{Streaming Random Forest}

\begin{columns}
 \begin{column}{0.5\textwidth}
  \begin{figure}[!ht]
  \center
  \includegraphics[width=\textwidth]{figures/srf_ap}
  \vspace{-0.25cm}
  \caption{Average Precision}
  \label{fig:srf_ap}
  \end{figure}
  \vspace{-0.5cm}
  \begin{figure}[!ht]
  \center
  \includegraphics[width=\textwidth]{figures/srf_iu}
  \vspace{-0.25cm}
  \caption{Intersection/Union}
  \label{fig:srf_iu}
  \end{figure}
 \end{column}
 
 \begin{column}{0.5\textwidth}
  Data:\\
  300 objects\\
  51 classes\\
  full revolution\\
  3 points of view \\
  \vspace{0.5cm}
  SRF - Streaming Random Forest\\
  ORF - Online Random Forest\\
  HT - Hoeffding Tree \\
 \end{column}
\end{columns}
\end{frame}


%---------------------------------------------------------------------------------------
\section{Discussion and Outlook}
\begin{frame}
\frametitle{Summary}
\begin{itemize}
 \item customized models of 3D enviornments
 \item fully interactive
 \item online and real time
 \item no pretraining
\end{itemize}

\end{frame}

\begin{frame}
\frametitle{Failures}
\begin{columns}
 \begin{column}{0.5\textwidth}
  \begin{figure}[!ht]	 
  \center
  \includegraphics[width=\textwidth]{figures/failures}
  \caption{Failure cases.}
  \label{fig:failures}
  \end{figure}
 \end{column}
 
 \begin{column}{0.5\textwidth}
  \begin{itemize}
   \item bleeding
   \item illumination change
   \item viewpoint change
  \end{itemize}
 \end{column}
\end{columns}

\end{frame}

\begin{frame}
\frametitle{Future Work}
\begin{itemize}
 \item class priors for different enviornments
 \item priors for class properties (vertical walls)
 \item discriminative geometrical features
 \item outdoor enviornments
 \item better scalability
\end{itemize}


\end{frame}


%---------------------------------------------------------------------------------------
\begin{frame}
\frametitle{References}
\fontsize{6pt}{7.2}\selectfont
\begin{itemize}
  \item Roberts, L. G. 1963. Machine perception of three-dimensional solids. Ph.D. thesis, Massachusetts Institute of Technology.

  \item Kim, B.-S. et. al. 2013. 3D scene understanding by voxel-CRF. In Proc. ICCV.

  \item Pradeep, V. et. al. 2013. Monofusion: Real-time 3D reconstruction of small scenes with a single web camera. In Proc. ISMAR.

  \item Herbst, E. et.al. 2014. Toward online 3-d object segmentation and mapping. In IEEE International Conference on Robotics and Automation (ICRA).

  \item Valentin, J. P. et. al. 2013. Mesh based semantic modelling for indoor and outdoor scenes. In Proc. CVPR.

  \item Salas-Moreno, R. F. et. al. 2013. SLAM++: Simultaneous localisation and mapping at the level of objects. In Proc. CVPR.

  \item Newcombe , R. A. et. al. 2011. KinectFusion: Real-time dense surface mapping and tracking. In Proc. ISMAR.  
  
  \item Curless , B. et. al. 1996. A volumetric method for building complex models from range images. In Proceedings of the 23rd annual conference on Computer graphics and interactive techniques. ACM, 303–312.
  
  \item Niessner , M. et. al. 2013. Real-time 3D reconstruction at scale using voxel hashing. ACM TOG 32, 6
\end{itemize}
\end{frame}


%---------------------------------------------------------------------------------------
\begin{frame}
\frametitle{References cont'd}
\fontsize{6pt}{7.2}\selectfont
\begin{itemize}
  \item Saffari , A. et. al. 2009. On-line random forests. In IEEE ICCV Workshop.
  
  \item Vitter , J. S. 1985. Random sampling with a reservoir. ACM TOMS 11, 1. 
  
  \item Lower , D. G. 1999. Object recognition from local scale-invariant features. In Proc. ICCV.
  
  \item Lafferty , J. et. al. 2001. Conditional random fields: Probabilistic models for segmenting and labeling sequence data.
  
  \item Ktahenbül, P. et. al. 2011. Efficient inference in fully connected CRFs with Gaussian edge potentials. In NIPS.

  \item Koller , D. et.al  , N. 2009. Probabilistic Graphical Models: Principles and Techniques. MIT Press
  
  \item Domingos, P. et. al. 2000. Mining high-speed data streams. In Proc. SIGKDD.
	
  \item Lai, K. et. al. 2011. A large-scale hierarchical multi-view rgb-d object dataset. In Proc. ICRA.
  
  \item Valentin, J. et. al. 2015. SemanticPaint: Interactive 3D Labeling and Learning at your Fingertips. SIGGRAPH.
\end{itemize}
\end{frame}


%---------------------------------------------------------------------------------------
\end{document}



